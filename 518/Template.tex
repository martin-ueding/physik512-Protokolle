\documentclass[11pt, ngerman, fleqn, DIV=15, headinclude, BCOR=2cm]{scrreprt}

\usepackage{../../header}

\usepackage{placeins}
\usepackage[maxfloats=50]{morefloats}

\usepackage{csquotes}

\usepackage{tikz}
\usetikzlibrary{chains}
\usetikzlibrary{shapes.geometric}

\tikzset{device/.style={
                rectangle,
                minimum size=6mm,
                draw=black
            },
            monitor/.style={
                rectangle,
                rounded corners=2mm,
                minimum size=6mm,
                draw=black
            },
        }

\usepackage{pgfplots}
\pgfplotsset{
    compat=1.9,
    width=0.8\linewidth,
    xticklabel style={/pgf/number format/use comma},
    yticklabel style={/pgf/number format/use comma},
}
\usepgfplotslibrary{polar}

\usepgfplotslibrary{external}
\tikzexternalize[mode=list and make]
\tikzsetexternalprefix{Abbildung-}

\DeclareSIUnit{\skt}{SKT}

\usepackage{booktabs}

\hypersetup{
    pdftitle=
}

\subject{Praktikumsprotokoll}
\title{Höhenstrahlung}
\subtitle{Versuch P518 -- Universität Bonn}
\author{
    Martin Ueding \\ \small{\href{mailto:mu@martin-ueding.de}{mu@martin-ueding.de}}
    \and
    Lino Lemmer \\
    \small{\href{mailto:l2@uni-bonn.de}{l2@uni-bonn.de}}
}

\date{\daterange{2014-07-02}{2014-07-03}}

\publishers{Tutor: Michael Lupberger}

\begin{document}

\maketitle

\begin{abstract}
    Wir vermessen die Energie- und Winkelabhängigkeit der Höhenstrahlung.
    Außerdem bestimmen wir die Lebensdauer der Myonen.
\end{abstract}

\tableofcontents

\chapter{Theorie}

\section{Höhenstrahlung}

Deutlich außerhalb der Atmosphäre befinden sich hochenergetische Teilchen, die
beim Eintritt in die Atmosphäre diverse Reaktionen verursachen. 

Die Winkelabhängigkeit ist, wenn man die Ablenkung durch das Erdmagnetfeld
ignoriert, isotrop in der Azimutrichtung und ungefähr $\cos(\theta)^2$ in der
Höhe. Diese Verteilung ist in Abbildung~\ref{fig:cos2} skizziert.

\begin{figure}[htbp]
    \centering
    \tikzsetnextfilename{cos2}
    \begin{tikzpicture}
        \begin{polaraxis}
            \addplot[black] table {cos2.csv};
        \end{polaraxis}
    \end{tikzpicture}
    \caption{%
        Erwartete Intensitätsverteilung der Höhenstrahlung $\cos(\theta)^2$.
        Dabei ist $\theta$ hier so gewählt, dass $\theta = \SI{0}{\degree}$
        einer Einstrahlung senkrecht zum Erdboden entspricht.
    }
    \label{fig:cos2}
\end{figure}

Die Primärstrahlung besteht zum Großteil aus Protonen und hat Energien bis
\SI{e21}{\electronvolt}, der Mittelwert liegt im Bereich
\SIrange{e9}{e10}{\electronvolt} \parencite[983]{meschede-gerthsen_24}.

Diese teils sehr harten Teilchen zerfallen in der Atmosphäre in Elektronen und
Positronen sowie harte Myonen. Myonen haben eine hohe Ruhemasse und verlieren
daher wenig Energie durch Bremsstrahlung. Daher kommen Sie hochenergetisch an
der Erdoberfläche an. \parencite[984]{meschede-gerthsen_24}

Dies sind die Teilchen, die wir hier im Versuch untersuchen werden.

\section{Standardmodell}

\subsection{Protonen und Neutronen}

Proton und Neutron sind Hadronen, die aus up- und down-Quarks zusammengesetzt
sind. Sie haben beide Spin $S$ und Isospin $I$ von $\frac12$. Sie unterscheiden
sich in der $I_3$-Komponente, dort hat das Proton $\frac12$, das Neutron
$-\frac12$.

Über das $W^-$-Boson, einem Vermittler der schwachen Wechselwirkung, kann das
Neutron in ein Proton zerfallen:
\[
    \mathrm n \quad\rightharpoonup\quad \mathrm p + \mathrm e^- + \bar
    \nuup_\mathrm e
    \qquad\iff\qquad
    \mathrm d
    \quad\rightharpoonup\quad
    \mathrm u + \mathrm W^-
    \quad\rightharpoonup\quad
    \mathrm u
    + \mathrm e^- + \bar \nuup_\mathrm e
\]

Das Proton ist stabil, das Neutron zerfällt jedoch innerhalb Minuten.

\subsection{Photonen}

Photonen sind die Vermittler der elektromagnetischen Wechselwirkung. Sie sind
masselos und tragen keinerlei Ladung. Sie haben jedoch Spin $S = 1$ und können
positive und negative Helizität haben, können also polarisiert sein. Die
Energie eines Photons ist $\hbar \omega$.

Gilt $\vnabla \vec E \neq 0$, so erlaubt es die Energie- und Impulserhaltung,
dass ein Photon ein Teilchen-Antiteilchen-Paar erzeugt (Paarbildung).

\subsection{Elektronen, Myonen, Neutrinos}

Elektronen und Myonen (sowie das Tau) sind die drei Familien von Leptonen im
Standardmodell. Myon und Tau sind letztlich schwerere Varianten des Elektrons.
Sie alle haben Ladung $-1$. Zu jedem dieser Teilchen gibt es ein Antiteilchen.
Außerdem gibt es zu jedem noch ein Neutrino und Antineutrino, diese werden als
$\nu_X$ für $X \in \{ \eup, \muup, \tauup \}$ bezeichnet. Auch diese haben
jeweils ein Antiteilchen. Neutrons wechselwirken nur durch die schwache
Wechselwirkung. Man geht davon aus, dass die Neutrinos eine kleine, allerdings
endliche Ruhemasse besitzen, da man Neutrinooszillation beobachtet.

Elektronen haben aufgrund ihrer geringen Ruhemasse einen relativ hohen
Energieverlust durch Bremsstrahlung. Die schwereren Teilchen haben nur eine
endliche Lebensdauer und gehen unter Abstrahlung von Neutrinos in den
Grundzustand, das Elektron, über.

\subsection{Leichte Mesonen}

% TODO Leichte Mesonen.

\section{Interaktion von Teilchen mit Materie}

\subsection{Bethe-Bloch-Gleichung}

% TODO

\subsection{Landau-Verteilung}

% TODO

\subsection{Schauerentwicklung}

% TODO

\section{Zerfallsgesetz und Lebensdauer}

Die Lebensdauer $\tau$ ist gerade die Zeit, bei der die Teilchenzahl auf
$\eup^{-1}$ gesunken ist. Das Zerfallsgesetz ist im Einklang damit durch
\[
    N(t) = N(0) \exp\del{- \frac t\tau}
\]
gegeben.

\section{Detektoren}

Um ionisierende Strahlung nachzuweisen gibt es den Geiger-Müller Zähler, in dem
die Strahlung Ionenpaare erzeugen, die durch eine Hochspannung abgesaugt und
registriert werden. In einer Nebel- oder Blasenkammer dienen Ionisationen als
Kondensationskeime, so können die Spuren verfolgt werden. Betreibt man eine
Diode in Sperrrichtung, erzeugt Strahlung wie beim Geier-Zähler Ionenpaar,
deren Absaugen einen kurzen Strompuls erzeugt.

In diesem Versuch benutzen wir Szintillationszähler. Dort deponieren Teilchen
ihre Energie durch Wechselwirkungen wie Compton- oder Photoeffekt,
Bremsstrahlung und Paarerzeugung. Das Szintillatormaterial wird dadurch
angeregt und strahlt Licht auf den Photomultiplier, der das Signal entsprechend
verstärkt.

Siehe auch §\ref{sec:frage1-1}.

\section{Logische Schaltungen}

\subsection{Constant Fraction Diskriminator (CFD)}
\label{ssec:CFD}

Ein Diskriminator ist ein Gerät, welches nur anspricht, wenn ein eintreffendes
Signal stärker ist, als ein bestimmter Grenzwert (engl. \emph{threshold}).
Trifft ein solches Signal ein, gibt der Diskriminator ein Standardsignal, zum
Beispiel ein Rechtecksignal aus.

Eine Anwendung des Diskriminators ist das Triggern: Trifft ein Signal ein, wird
ein neues, standardisiertes abgegeben. So ist zum Beispiel der zeitliche
Abstand zwischen zwei eintreffenden Signalen zu messen. Wann genau der
Diskriminator triggert, kommt auf die Art an.

Ein CFD triggert, wenn ein bestimmter Anteil der Maximalamplitude erreicht
wird. Dazu wird das einkommende Signal gesplittet, ein Teil wird zeitlich
verzögert, der andere invertiert und um einen Faktor $k$ gedämpft. Anschließend
werden beide Signale wieder addiert. Ein vorher rein positives Signal erhält so
eine Nullstelle, welche durch $k$ bestimmt wird und den Triggerpunkt definiert.

\parencite{Ueding/525}

\subsection{Koinzidenz}

Eine Koinzidenzeinheit wirkt als logisches Und. Sie hat mehrere Eingänge und
einen Ausgang. Genau dann, wenn an allen Eingängen der Wert 1 anliegt, steht
der Ausgang auf 1.

\subsection{FPGA}

Ein Field Programmable Gate Array (FPGA) ist ein Gerät, das frei verschaltbare
Gatter bietet. Die einzelnen Gatter können als verschiedene Gatter programmiert
werden (Und, Oder, exklusives Oder, Nicht-Und, …) und im Rahmen der
Anordnungsmöglichkeiten zu beliebigen Logikschaltungen kombiniert werden.
Dadurch ist es möglich, eine Schaltung zu programmieren.

Der Vorteil von einem FPGA gegenüber einen kleinen Computer ist, dass das FPGA
mit einer festen Schaltung deutlich weniger Zyklen als ein Computer mit einem
Programm braucht. Daher werden FPGAs direkt hinter Detektoren eingesetzt, um
mit den großen Zählraten klar zu kommen.

\section{Vorbereitungsfragen}

\subsection{Vorbereitungsfragen zur Winkelverteilung}

Dies sind die Vorbereitungsfragen aus \parencite[11]{physik512-Anleitung}.

\subsubsection{Frage 1}
\label{sec:frage1-1}

\begin{quote}
    Wie funktionieren Szintillatoren und Photomultiplier? Welche physikalischen
    Prozesse und apparativen Einflüsse bestimmen den zeitlichen Verlauf des
    Photomultiplier-Ausgangssignals? Wie hängt die Pulshöhe von der gewählten
    Hochspannung am Photomultiplier ab?
\end{quote}

In einem Szintillator verlieren eingehende Teilchen und Photonen durch
verschiedene Prozesse (Comptoneffekt, Photoeffekt, Paarbildung) ihre Energie
und wandeln sie letztendlich in Photonen um. Diese treffen auf den
Photomultiplier, wo sie Elektronen auslösen. Diese Elektronen werden durch
starke elektrische Felder zu der nächsten Anode beschleunigt. Dort schlagen sie
weitere Elektronen heraus, es entwickelt sich eine Lawine.

Der zeitliche Verlauf wird durch die Vorgänge im Szintillator beeinflusst. Je
nach Material dauert es unterschiedlich lange, bis die angeregten Atome im
Material in den Grundzustand übergehen und ihre Energie abgeben.

% TODO Hier vielleicht noch etwas mehr schreiben?

Je größer die angelegte Hochspannung ist, desto stärker werden die Elektronen
beschleunigt und desto mehr können sie in jedem Schritt herausschlagen. Jedoch
kann die Stromversorgung nicht mehr schnell genug Elektronen nachliefern, so
dass eine Sättigung eintritt. Somit sollte sich ein linearer Verlauf mit
Plateau einstellen.

\subsubsection{Frage 2}

\begin{quote}
    Wie funktionieren Diskriminator- und Koinzidenz-Schaltungen? Wie sehen die
    jeweiligen Ausgangssignale aus? Wie beeinflusst die Schwellenhöhe eines
    Diskriminators die zeitliche Lage des Ausgangssignals gegenüber dem
    Eingangssignal?
\end{quote}

Zur Funktionsweise des Diskriminators siehe §\ref{ssec:CFD}. Eine
Koinzidenzschaltungen benutzt ein logisches Und.

Der Ausgang von einem Diskriminator ist ein Rechteckpuls, das über dem
ursprünglichen Signal liegt. Die zeitliche Lage sowie die Länge des Pulses
hängt vom Diskriminator ab. Ein Oszillogramm ist in
Abbildung~\ref{fig:oszi-cfd} gezeigt.

\begin{figure}[htbp]
    \centering
    \includegraphics[width=.5\linewidth]{../Oszi-CFD.pdf}
    \caption{%
        Slow-Signal eines Photomultipliers zusammen mit dem
        Signal des Einkanalanalysators (SCA). Bild aus
        \parencite[Abbildung~2.8]{Ueding/525}. Ein Diskriminator ist zwar kein
        Einkanalanalysator, jedoch ist das Ausgangsbild vergleichbar.
    }
    \label{fig:oszi-cfd}
\end{figure}

Die Koinzidenzeinheit wird aus zwei digitalen Rechteckpulsen das Produkt
erstellen, also wieder einen Rechteckpulse, der nur in der überlappenden Region
1 ist. In Abbildung~\ref{fig:oszi-koinzidenz} ist das Oszillogramm von zwei
Eingängen gezeigt. Nach der Koinzidenz ist nur ein etwas schmalerer
Rechteckpuls zu sehen.

\begin{figure}[htbp]
    \centering
    \includegraphics[width=.5\linewidth]{../Oszi-Koinzidenz.pdf}
    \caption{%
        Oszillogramm der Slow-Zweige von zwei verschiedenen Photomultiplier. Es
        wurde auf eins der Signale getriggert. Bild aus
        \parencite[Abbildung~2.24]{Ueding/525}.
    }
    \label{fig:oszi-koinzidenz}
\end{figure}

% TODO Antwort einfügen.

\subsubsection{Frage 3}

\begin{quote}
    Die Zählrate für jeden einzelnen Zähler betrage etwa 1 Hz am
    Diskriminatorausgang. Wie groß ist die theoretisch zu erwartende Rate der
    Zufallskoinzidenzen für eine der Dreifach- Koinzidenz-Schaltungen zur
    Messung der Winkelverteilung der Höhenstrahlung? Und für die
    Koinzidenz-Schaltung, die zur Messung des Pulshöhenspektrums verwendet
    wird? Wie groß sind jeweils die Totzeiten? Machen Sie einen Vorschlag, wie
    die Rate der Zufallskoinzidenzen für eine der
    Dreifach-Koinzidenz-Schaltungen gemessen werden kann.
\end{quote}

% TODO Antwort einfügen.

\subsubsection{Frage 4}

\begin{quote}
    Welchen inhaltlichen Zusammenhang haben Bethe-Bloch-Gleichung und
    Landau-Verteilung?
\end{quote}

% TODO Antwort einfügen.

\subsubsection{Frage 5}

\begin{quote}
    Welche Beiträge hat das Pulshöhenspektrum? Welche physikalischen und
    apparativen Einflüsse bestimmen die Form des Pulshöhenspektrums? Wie
    unterscheiden sich zwei Pulshöhenspektren, wenn diese mit und ohne aktiven
    Gate-Eingang aufgenommen werden?
\end{quote}

% TODO Antwort einfügen.

\subsubsection{Frage 6}

\begin{quote}
    Wie hängen Pulshöhenspektrum und Schwellenkurve zusammen? Wie ändert sich
    die Form der Schwellenkurve, wenn man die Anzahl der Koinzidenzsignale
    anstelle der Diskriminatorsignale betrachtet?
\end{quote}

% TODO Antwort einfügen.

\subsubsection{Frage 7}

\begin{quote}
    Welche Funktion erfüllt das in Abbildung 5 gezeigte LabVIEW-Programm?
\end{quote}

Wir haben das Programm mit einigen Anmerkungen versehen, siehe
Abbildung~\ref{fig:labview-test}.

\begin{figure}[htbp]
    \centering
    \includegraphics[width=\linewidth]{../Programm-crop.pdf}
    \caption{%
        LabVIEW Programm aus \parencite[Abbildung~5]{physik512-Anleitung}.
    }
    \label{fig:labview-test}
\end{figure}

Dieses Programm realisiert eine Zählratenbestimmung mit einem Zähler mit
Diskriminator und Gate. Die Bauteile im Detail:

\paragraph{Erstes Bild: Reset}

Im ersten Bild des Filmstreifens wird die NIMBox zurückgesetzt und
initialisiert.

\paragraph{Zweites Bild: Init}

Hier wird die Verbindung zur NIMBox aufgebaut und die USB Verbindung als handle
(dunkelviolett) zum nächsten Bild gegeben. Mit diesem handle können alle Module
auf die richtige NIMBox bezogen werden.

\paragraph{Drittes Bild: Setup}

Oben ist das violette handle für die USB-Verbindung zum nächsten Bild
geschleift. Dieses handle ist ebenfalls an alle Bauteile angeschlossen.

Das grüne ist ein Diskriminator. Dieser bekommt von oben eine 1, so dass dies
der erste Diskriminator in der Box ist. Die drei weiteren Eingänge, $-420$, 20
und 5 stehen für die Schwelle, Hystere in \si{\milli\volt} bzw. für die Dauer
des Ausgangspulses in \SI{10}{\nano\second}. Das „Connect“ führt hier dazu,
dass dieser Diskriminator so konfiguriert wird.

Das türkise Element ist für digitale Ein- und Ausgabe (digitale IO). Auch
dieses ist das erste in der Box. Mit einem „Connect“ wird es ebenfalls
angeschlossen. Der Cluster mit Parametern gibt noch an, dass NIM anstelle von
TTL als Übertragunsstandard gewählt worden ist. Dies muss mit der Position des
physischen Jumpers übereinstimmen. Dieses Modul wird dazu benutzt, digitale
Signale in das Programm zu bekommen.

Die Signale von Diskriminator und digitaler Eingabe werden zusammen auf einen
Koinzidenzzähler (orange) gelegt. Dies ist der erste in der Box und wird auch
so angeschlossen.

Dessen Ausgang geht wiederum auf eine LED, diesmal die dritte in der Box.

\paragraph{Viertes Bild: Schleife}

Auch im vierten Bild kommt das handle für die USB-Verbindung an und wird
weitergegeben. Das handle wird in die Endlosschleife hineingegeben. Dort wird
der erste Koinzidenzzähler (dunkles orange) ausgelesen und durch die bis
dorthin verstrichene Zeit geteilt. Diese Zeit liefert ein Zeitgeber (hellblau).
In der Ausgabe erscheint die Rate sowie die Zeit seit Start. Um die Schleife
verlassen zu können ist ein Schalter als Abbruchbedingung in die Vorderansicht
eingebaut.

\paragraph{Fünftes Bild: Ende}

Zuletzt wird die USB-Verbindung getrennt.

\subsection{Vorbereitungsfragen zur Myonenlebensdauer}

Dies sind die Vorbereitungsfragen aus \parencite[14]{physik512-Anleitung}.

\subsubsection{Frage 1}

\begin{quote}
    Es kommen sowohl Myonen als auch Antimyonen auf der Erdoberfläche an.
    Welcher atomphysikalische Prozeß ist für Myonen möglich, aber nicht für
    Antimyonen? Wie beeinflußt das qualitativ die in diesem Versuch gemessene
    Lebensdauer?
\end{quote}

% TODO Antwort einfügen.

\subsubsection{Frage 2}

\begin{quote}
    Machen Sie einen Vorschlag, wie man unter Verwendung von Diskriminatoren,
    Verzögerungs- kabeln und Koinzidenz-Schaltungen die START-und STOP-Impulse
    erzeugen kann.
\end{quote}

% TODO Antwort einfügen.

\subsubsection{Frage 3}

\begin{quote}
    Entwerfen Sie das Blockschaltbild für Messkreis und Monitorkreis.
\end{quote}

% TODO Antwort einfügen.

\subsubsection{Frage 4}

\begin{quote}
    Können Sie eine andere Gestaltung der Zeitintervalle vorschlagen, die die
    Genauigkeit der Myonlebensdauerbestimmung optimieren würde?
\end{quote}

Die Intervalleinteilung in der Anleitung ist linear. Da ein exponentieller
Abfall in der Kurve zu erwarten ist, werden in den ersten Intervallen mehr
Ereignisse gemessen als in den letzten. Dadurch ist der relative Fehler in den
ersten Intervallen kleiner. Verkleinert man die ersten und streckt dafür die
letzten Intervalle exponentiell, kann man dadurch den relativen Fehler in allen
Intervallen gleich bekommen.

\subsubsection{Frage 5}

\begin{quote}
    Das aus der Apparatur kommende Start-Signal wird gegenüber dem Stop-Signal
    um ca. 100 ns verzögert. Warum ist das notwendig? Wie wirkt es sich auf das
    Messergebnis aus?
\end{quote}

Diese Verzögerung wurde auch schon in Versuch 525 benutzt, um sicherzustellen,
dass das Start-Signal wirklich vor dem Stop-Signal kommt. Das Messergebnis wird
dadurch nur zu größeren Zeiten verschoben, Die Lebenszeit $\tau$ der Myonen
erhalten wir jedoch aus dem exponentiellen Abfall der Zählrate. Diese
Verschiebung ändert daran nichts.

\subsubsection{Frage 6}

\begin{quote}
    Wie können scheinbare Myonzerfälle zustandekommen? Welche Messgrößen
    braucht man, um die erwartete Anzahl von Zufallsereignissen berechnen zu
    können?
\end{quote}

% TODO Antwort einfügen.

\chapter{Winkelverteilung}

Um die Übersicht über die vielen kleinen Schritte zu gewährleisten, möchten wir
hier erst einmal die Schritte grob zusammenstellen:

\begin{enumerate}
    \item
        Finden der optimalen Verzögerung der Signale, die in die Koninzidenz
        gehen, damit deren Überschneidung maximiert wird. Siehe
        §\ref{sec:optimieren_verzoegerung}.

    \item
        Einstellung der Diskriminatorschwellen um den Untergrund zu filtern.
        Die zugehörige Messung läuft über Nacht. Siehe
        §\ref{sec:einstellung_diskriminatorschwelle}.

    \item
        Aufnahme der Winkelverteilung, also Zählrate über alle Energien gegen
        Winkel. Diese Messung läuft über Wochen. Siehe
        §\ref{sec:langzeit_winkel}
        
    \item
        Aufnahme eines Pulshöhenspektrums der Höhenstrahlung, also letztlich
        Zählrate gegen Energie. Dies läuft ebenfalls über Wochen. Siehe
        §\ref{sec:langzeit_puls}
\end{enumerate}

\section{Aufbau, Geräte}

\subsection{Zählerring}

Das Detektorsystem besteht aus 24 ringförmig angeordneten
Szintillationszählern, die um einen zentralen Zähler angeordnet sind. Der ganze
Ring steht senkrecht, so dass die Winkelverteilung im Bezug auf den Zenit
bestimmt werden kann.

% TODO Vielleicht ein entsprechendes Bild hier einfügen?

\subsection{Elektronik}

\subsubsection{Diskriminatoren}

An jeden Detektor ist ein Diskriminator angeschlossen. Damit werden die
analogen Pulse in digitale Pulse umgewandelt.

\subsubsection{Verteiler}

Der mittlere Zähler Z25 wird für die Koinzidenz von allen Detektorpaaren
benötigt. Daher wird dessen Signal nach dem Diskriminator durch einen Fanout
zwölffach aufgefächert.

\subsubsection{Koinzidenz}

Ziel der Koinzidenz ist, nur solche Ereignisse zu filtern, bei denen ein
Teilchen genau durch durch die Mitte und ein gegenüberliegendes Detektorpaar
fliegt. Um dies zu filtern, gehen die digitalen Signale von zwei
gegenüberliegenden Detektoren sowie des mittleren Detektors in eine
Koinzidenzeinheit.

\begin{figure}[htbp]
    \centering
    \tikzsetnextfilename{koinzidenz}
    \begin{tikzpicture}

        %< for i in range(1, 25): >%
        \node[draw, circle] (Z<< i >>) at (<< 90 - i * 15 >>:7) {Z<< i >>};
        \node[draw, rectangle] (D<< i >>) at (<< 90 - i * 15 >>:5) {D<< i >>};
        \draw[->] (Z<< i >>) -- (D<< i >>);
        %< endfor >%

        \node[draw, circle] (Z25) at (0, 0) {Z25};

        \node[draw, rectangle] (D25) [below=of Z25] {D25};

        \node[draw, rectangle] (fan) [below=of D25] {Fanout};

        \node[draw, rectangle] (con1) [right=of Z25] {Koinzidenz};
        \node[draw, rectangle] (con2) [left=of Z25] {Koinzidenz};

        \begin{scope}[->]
            \draw (Z25) -- (D25);
        \end{scope}

        \begin{scope}[->, dashed]
            \draw (D25) -- (fan);
            
            \draw (fan) -- (con1);
            \draw (fan) -- (con2);

            \draw (D2) -- (con2);
            \draw (D14) -- (con2);

            \draw (D22) -- (con1);
            \draw (D10) -- (con1);
        \end{scope}

        \draw[->] (45:8) -- ++(0, 1) node[right, midway] {$\theta = 0$};

    \end{tikzpicture}
    \caption{%
        Anordnung der Detektoren. Zur Übersicht sind nur zwei
        Koinzidenzschaltungen eingezeichnet. Durchgezogene und gestrichelte
        Linien stehen für analoge bzw. digitale Signalübertragungen.
    }
    \label{fig:koinzidenz}
\end{figure}

\subsubsection{Zähler}

\section{Justierung}

\subsection{Optimieren der Verzögerung}
\label{sec:optimieren_verzoegerung}

In LabVIEW konstruieren wir ein logisches oder zwischen den Diskriminatoren D23,
D24, D1 und D2. Mit dessen Ausgang konstruieren wir eine Koinzidenz mit dem
Ausgang von D12.

\subsection{Einstellung der Diskriminatorschwelle}
\label{sec:einstellung_diskriminatorschwelle}

\subsubsection{Entwicklung der Schaltung}

Zur Ermittlung der optimalen Diskriminatorschwelle müssen wir eine
Schwellenkurve aufnehmen. Dies ist eine Kurve mit Zählrate gegen
Diskriminatorschwelle. Da wir von einer recht monoenergetischen Myonenstrahlung
ausgehen, erwarten wir eine Stufenfunktion mit einer Stufe. Bei kleinen
Schwellen werden wir den kompletten Untergrund mitnehmen. Ab der Schwelle, die
der Myonenstrahlung entspricht, erwarten wir einen Einbruch. Die
Diskriminatorschwelle wählen wir anschließend so, dass wir den Untergrund
filtern können, jedoch möglichst wenig von den Myonen verlieren.

Dazu konstruieren wir in LabVIEW eine Schaltung, die über Nacht die
verschiedenen Schwellen durchfährt und die Zählraten misst. Dabei müssen
folgende Daten für die weitere Durchführung aufgezeichet werden.

\begin{itemize}
    \item
        Gewählte Schwelle für Z12

    \item
        Anzahl Ereignisse in D12

    \item
        Anzahl der oder-Ereignisse

    \item
        Anzahl der Ereignisse in Z25

    \item
        Messdauer

    \item
        Anzahl der Koinzidenzen
\end{itemize}

Im Vorfeld haben wir uns überlegt, wie diese Programme grob aussehen müssten.
In Abbildung~\ref{fig:entwurf-1} ist der Teil, der die Koinzidenz
implementiert, gezeigt. Dort werden die Signale aus den Diskriminatoren über
drei oder-Gatter zusammengeführt. Zusammen mit dem Signal aus D12 gehen diese
dann in den Koinzidenzzähler. Dessen Logikausgang geben wir zur Kontrolle auf
eine LED sowie durch die Lemo-Buchse nach draußen.

\begin{figure}[htbp]
    \centering
    \includegraphics[width=\linewidth]{../Drawing-0001.pdf}
    \caption{%
        Entwurf einer LabVIEW Schaltung für die Koinzidenz.
    }
    \label{fig:entwurf-1}
\end{figure}

Um nun die verschiedenen Schwellen durchzumessen, müssen wir eine Schleife
konstruieren und die verschiedenen Schwellen eine gewisse Zeit lang messen.
Dazu halten wir eine \texttt{for}-Schleife für sinnvoll. Aus dem Index
errechnen wir die Schwelle und sammeln alle interessanten Daten in Arrays. Mit
Abbildung~\ref{fig:entwurf-2} haben wir unsere Skizze eingefügt.

Die Schwellenzahl, die Messdauer und der Schwellenschritt müssen noch
eingetragen werden. Dabei müssen wir darauf achten, dass die Messzeit bis zum
nächsten Morgen nicht überschritten wird.

In jedem Schleifendurchlauf wird die Schwelle berechnet und der Diskriminator
entsprechend eingestellt. Die Messzeit wird gewartet, bis entsprechend viele
Ereignisse im Koinzidenzzähler angekommen sind. Mit weiteren Zählern zählen wir
noch weitere Größen, die interessant werden können. Diese Werte geben wir durch
den „loop tunnel” in Arrays weiter.

\begin{figure}[htbp]
    \centering
    \includegraphics[width=\linewidth]{../Drawing-0002.pdf}
    \caption{%
        Entwurf einer LabVIEW Schaltung für das Durchmessen der Schwellen.
    }
    \label{fig:entwurf-2}
\end{figure}

Zuletzt speichern wir die Daten aus den Arrays ab. Dazu bündeln wir sie in ein
zweidimensionales Array und speichern es als Textdatei ab. Dies ist in
Abbildung~\ref{fig:entwurf-3} dargestellt.

\begin{figure}[htbp]
    \centering
    \includegraphics[width=.8\linewidth]{../Drawing-0003.pdf}
    \caption{%
        Entwurf einer LabVIEW Schaltung für das Speichern der Daten.
    }
    \label{fig:entwurf-3}
\end{figure}

% TODO Wir stellen das ganze exemplarisch für Z12 ein. Was müssen wir mit den
% anderen Zählern machen? Sind die (a) schon fertig eingestellt, (b) erhalten
% die gleichen Parameter wie Z12 oder (c) müssen jeweils noch ausgemessen
% werden?

\subsubsection{Auswertung der Daten}

\section{Langzeitmessung zur Winkelverteilung}
\label{sec:langzeit_winkel}

\section{Langzeitmessung des Pulshöhenspektrums}
\label{sec:langzeit_puls}

\chapter{Myonenlebensdauer}

\section{Aufbau}

\subsection{Monitorkreis}

\section{Durchführung}

\section{Auswertung}

\chapter{Ergebnis}

\end{document}

% vim: spell spelllang=de tw=79
