\input{../../header.tex}

\usepackage[section]{placeins}

\usepackage{csquotes}

\usepackage{tikz}
\usetikzlibrary{chains}
\usetikzlibrary{shapes.geometric}

\tikzset{device/.style={
                rectangle,
                minimum size=6mm,
                draw=black
            },
            monitor/.style={
                rectangle,
                rounded corners=2mm,
                minimum size=6mm,
                draw=black
            },
        }

\usepackage{pgfplots}
\pgfplotsset{
    compat=1.5,
    width=0.8\linewidth,
    xticklabel style={/pgf/number format/use comma},
    yticklabel style={/pgf/number format/use comma},
}

\usepgfplotslibrary{external}
\tikzexternalize

\usepackage{booktabs}

\hypersetup{
    pdftitle=
}

\subject{Praktikumsprotokoll}
\title{$\boldsymbol\beta$-Spektrometer}
\subtitle{Versuch P523 -- Universität Bonn}
\author{
    Martin Ueding \\ \small{\href{mailto:mu@martin-ueding.de}{mu@martin-ueding.de}}
    \and
    Lino Lemmer \\
    \small{\href{mailto:l2@uni-bonn.de}{l2@uni-bonn.de}}
}

\date{\daterange{2014-04-24}{2014-04-25}}
\publishers{Tutor: TODO}

\begin{document}

\maketitle

\begin{abstract}
\end{abstract}

\tableofcontents

\chapter{Theorie}

\section{Zerfallsarten}

\newcommand\betaplus{\betaup^+}
\newcommand\betaminus{\betaup^-}

Instabile Kerne können über verschiedene Kanäle zerfallen. Die für diesen
Versuch Interessanten sind die $\betaplus$- und $\betaminus$-Zerfälle sowie der
Elektroneneinfang.

Bei $\betaminus$-Zerfall passiert auf Quarkebene folgendes:
\[
    \mathrm d
    \quad\rightharpoonup\quad
    \mathrm u + \mathrm W^-
    \quad\rightharpoonup\quad
    \mathrm u + \mathrm e^- + \bar\nuup_{\mathrm e}.
\]

Auf Nukleonenebene ohne Zwischenschritt sieht der Vorgang wie folgt aus:
\[
    \mathrm n
    \quad\rightharpoonup\quad
    \mathrm p + \mathrm e^- + \bar\nuup_{\mathrm e}..
\]

Analog funktioniert der $\betaplus$-Zerfall, bei dem ein $\mathrm W^+$-Boson
ausgetauscht wird:
\[
    \mathrm p
    \quad\rightharpoonup\quad
    \mathrm n + \mathrm e^+ + \nuup_{\mathrm e}.
\]

Der Elektroneneinfang ist ein $\betaplus$-Zerfall, bei dem das Position auf der
anderen Seite der Reaktionsgleichung steht:
\[
    \mathrm p + \mathrm e^-
    \quad\rightharpoonup\quad
    \mathrm n + \nuup_{\mathrm e}.
\]

Der Kern, dessen Ladungszahl um eins verändert worden ist, wird meist in einem
angeregten Zustand hinterlassen. Diese Energie wird durch $\gammaup$-Strahlung
abgebaut. Es ist möglich, dass die Energie von einem Hüllenelektron absorbiert
wird, welches aufgrund der großen Energie vom Kern getrennt wird. Diese
Elektronen nennt man \emph{Augerelektronen}.

Nach einem Elektroneneinfang ist ein Zustand mit niedriger Energie in der Hülle
unbesetzt. Elektronen aus höheren Zuständen werden in die niedrigeren Zustände
fallen und dabei charakteristische Röntgenstrahlung emittieren.

\section{Fermitheorie}

\section{Spektrometeraufbau}

\section{Detektoren}

\section{Hystere}

\chapter{Durchführung}

\chapter{Auswertung}


\IfFileExists{\bibliographyfile}{
    \printbibliography
}{}

\end{document}

% vim: spell spelllang=de tw=79
