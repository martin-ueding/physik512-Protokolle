\input{../../header.tex}

\usepackage[section]{placeins}

\usepackage{csquotes}

\usepackage{tikz}
\usetikzlibrary{chains}
\usetikzlibrary{shapes.geometric}

\tikzset{device/.style={
                rectangle,
                minimum size=6mm,
                draw=black
            },
            monitor/.style={
                rectangle,
                rounded corners=2mm,
                minimum size=6mm,
                draw=black
            },
        }

\usepackage{pgfplots}
\pgfplotsset{
    compat=1.5,
    width=0.8\linewidth,
    xticklabel style={/pgf/number format/use comma},
    yticklabel style={/pgf/number format/use comma},
}

\usepgfplotslibrary{external}
\tikzexternalize

\usepackage{booktabs}

\hypersetup{
    pdftitle=
}

\subject{Praktikumsprotokoll}
\title{$\boldsymbol\betaup$-Spektrometer}
\subtitle{Versuch P523 -- Universität Bonn}
\author{
    Martin Ueding \\ \small{\href{mailto:mu@martin-ueding.de}{mu@martin-ueding.de}}
    \and
    Lino Lemmer \\
    \small{\href{mailto:l2@uni-bonn.de}{l2@uni-bonn.de}}
}

\date{\daterange{2014-04-24}{2014-04-25}}
\publishers{Tutor: TODO}

\begin{document}

\maketitle

\nocite{Hof/Poltergeist}
\nocite{physik512-Anleitung}
\nocite{Riezler/Kernphysikalisches}

\begin{abstract}
\end{abstract}

\tableofcontents

\chapter{Theorie}

\section{Zerfallsarten}

\newcommand\betaplus{\betaup^+}
\newcommand\betaminus{\betaup^-}

Instabile Kerne können über verschiedene Kanäle zerfallen. Die für diesen
Versuch Interessanten sind die $\betaplus$- und $\betaminus$-Zerfälle sowie der
Elektroneneinfang.

Bei $\betaminus$-Zerfall passiert auf Quarkebene folgendes:
\[
    \mathrm d
    \quad\rightharpoonup\quad
    \mathrm u + \mathrm W^-
    \quad\rightharpoonup\quad
    \mathrm u + \mathrm e^- + \bar\nuup_{\mathrm e}.
\]

Auf Nukleonenebene ohne Zwischenschritt sieht der Vorgang wie folgt aus:
\[
    \mathrm n
    \quad\rightharpoonup\quad
    \mathrm p + \mathrm e^- + \bar\nuup_{\mathrm e}..
\]

Analog funktioniert der $\betaplus$-Zerfall, bei dem ein $\mathrm W^+$-Boson
ausgetauscht wird:
\[
    \mathrm p
    \quad\rightharpoonup\quad
    \mathrm n + \mathrm e^+ + \nuup_{\mathrm e}.
\]

Der Elektroneneinfang ist ein $\betaplus$-Zerfall, bei dem das Position auf der
anderen Seite der Reaktionsgleichung steht:
\[
    \mathrm p + \mathrm e^-
    \quad\rightharpoonup\quad
    \mathrm n + \nuup_{\mathrm e}.
\]

Der Kern, dessen Ladungszahl um eins verändert worden ist, wird meist in einem
angeregten Zustand hinterlassen. Diese Energie wird durch $\gammaup$-Strahlung
abgebaut.

Nach einem Elektroneneinfang ist ein Zustand mit niedriger Energie in der Hülle
unbesetzt. Elektronen aus höheren Zuständen werden in die niedrigeren Zustände
fallen und dabei charakteristische Röntgenstrahlung emittieren.

Es ist möglich, dass die Energie von einem Hüllenelektron absorbiert
wird, welches aufgrund der großen Energie vom Kern getrennt wird. Diese
Elektronen nennt man \emph{Augerelektronen}.

\section{Fermitheorie}

\subsection{Kuriedarstellung}

In der Kuriedarstellung\footnote{Nach
F.\,N.\,D\@.~Kurie. \parencite{wikipedia/Kurie}} trägt man nicht die Zählrate
gegen die kinetische Energie auf, sondern transformiert die Zählrate zur Größe
$K$ wie folgt: \parencite[(10.6)]{Kolanoski/Beta}
\[
    K(E_{\text e})
    = \sqrt{
        \dpd{N}{p_{\text e}} \frac{1}{p_{\text e}^2 F(p_{\text e}, Z)}
    }
    \propto
    \mathcal M_{\text i, \text f} \cdot (E_0 - E_{\text e}).
\]

Energie und Impuls können durch folgende dimensionslose Größen ausgedrückt
werden: \parencite[§1.5]{Guenther/K124}
\[
    \epsilon := \frac{E}{m_{\text e} c^2}
    \eqnsep
    \eta := \frac{p_{\text e}}{m_{\text e} c}.
\]

\section{Spektrometeraufbau}

\section{Detektoren}

\section{Hystere}

Eisen ist ferromagnetisch, es behält seine Magnetisierung $M$ auch ohne
angelegtes Magnetfeld $H$ bei. Legt man Eisen in ein Magnetfeld und erhöht die
Flussdichte bis zu einem Maximum (so dass die Magnetisierung in Sättigung geht)
und erniedrigt wieder bis $H = 0$, so nennt man die verbleibende Magnetisierung
\emph{Remanenzfeld}. Dieser Effekt ist isotrop, so dass man durch den
Mittelwert der beiden Remanenzfeldstärken die neutrale Mitte herausfinden kann.

Möchte man einen Ferromagneten entmagnetisieren, muss man das Magnetfeld, indem
sich der Ferromagnet befindet, in mehreren Durchläufen zwischen dem positiven
und negativen Maximum regeln, wobei ab dem zweiten Durchlauf der
Betrag der maximalen Magnetfeldstärke reduziert wird. Dadurch wird das
Remanenzfeld bei jedem Durchlauf schwächer, bis am Ende verschwindet. Diesen
Vorgang nennt man \emph{Hysterese}.

\chapter{Durchführung}

\chapter{Auswertung}


\IfFileExists{\bibliographyfile}{
    \printbibliography
}{}

\end{document}

% vim: spell spelllang=de tw=79
