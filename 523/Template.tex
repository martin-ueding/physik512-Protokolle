\input{../../header.tex}

\usepackage[section]{placeins}

\usepackage{csquotes}

\usepackage{tikz}
\usetikzlibrary{chains}
\usetikzlibrary{shapes.geometric}

\tikzset{device/.style={
                rectangle,
                minimum size=6mm,
                draw=black
            },
            monitor/.style={
                rectangle,
                rounded corners=2mm,
                minimum size=6mm,
                draw=black
            },
        }

\usepackage{pgfplots}
\pgfplotsset{
    compat=1.5,
    width=0.8\linewidth,
    xticklabel style={/pgf/number format/use comma},
    yticklabel style={/pgf/number format/use comma},
}

\usepgfplotslibrary{external}
\tikzexternalize

\usepackage{booktabs}

\hypersetup{
    pdftitle=
}

\subject{Praktikumsprotokoll}
\title{$\boldsymbol\betaup$-Spektrometer}
\subtitle{Versuch P523 -- Universität Bonn}
\author{
    Martin Ueding \\ \small{\href{mailto:mu@martin-ueding.de}{mu@martin-ueding.de}}
    \and
    Lino Lemmer \\
    \small{\href{mailto:l2@uni-bonn.de}{l2@uni-bonn.de}}
}

\date{\daterange{2014-04-24}{2014-04-25}}
\publishers{Tutor: TODO}

\begin{document}

\maketitle

\nocite{Hof/Poltergeist}

\begin{abstract}
\end{abstract}

\tableofcontents

\chapter{Theorie}

\section{Zerfallsarten}

\newcommand\betaplus{\betaup^+}
\newcommand\betaminus{\betaup^-}

Instabile Kerne können über verschiedene Kanäle zerfallen. Die für diesen
Versuch Interessanten sind die $\betaplus$- und $\betaminus$-Zerfälle sowie der
Elektroneneinfang.

Bei $\betaminus$-Zerfall passiert auf Quarkebene folgendes:
\[
    \mathrm d
    \quad\rightharpoonup\quad
    \mathrm u + \mathrm W^-
    \quad\rightharpoonup\quad
    \mathrm u + \mathrm e^- + \bar\nuup_{\mathrm e}.
\]

Auf Nukleonenebene ohne Zwischenschritt sieht der Vorgang wie folgt aus:
\[
    \mathrm n
    \quad\rightharpoonup\quad
    \mathrm p + \mathrm e^- + \bar\nuup_{\mathrm e}..
\]

Analog funktioniert der $\betaplus$-Zerfall, bei dem ein $\mathrm W^+$-Boson
ausgetauscht wird:
\[
    \mathrm p
    \quad\rightharpoonup\quad
    \mathrm n + \mathrm e^+ + \nuup_{\mathrm e}.
\]

Der Elektroneneinfang ist ein $\betaplus$-Zerfall, bei dem das Position auf der
anderen Seite der Reaktionsgleichung steht:
\[
    \mathrm p + \mathrm e^-
    \quad\rightharpoonup\quad
    \mathrm n + \nuup_{\mathrm e}.
\]

Der Kern, dessen Ladungszahl um eins verändert worden ist, wird meist in einem
angeregten Zustand hinterlassen. Diese Energie wird durch $\gammaup$-Strahlung
abgebaut.

Nach einem Elektroneneinfang ist ein Zustand mit niedriger Energie in der Hülle
unbesetzt. Elektronen aus höheren Zuständen werden in die niedrigeren Zustände
fallen und dabei charakteristische Röntgenstrahlung emittieren.

Es ist möglich, dass die Energie von einem Hüllenelektron absorbiert
wird, welches aufgrund der großen Energie vom Kern getrennt wird. Diese
Elektronen nennt man \emph{Augerelektronen}.

\section{Fermitheorie}

\subsection{Kuriedarstellung}

In der Kuriedarstellung\footnote{Nach
F.\,N.\,D\@.~Kurie. \parencite{wikipedia/Kurie}} trägt man nicht die Zählrate
gegen die kinetische Energie auf, sondern transformiert die Zählrate zur Größe
$K$ wie folgt: \parencite[(10.6)]{Kolanoski/Beta}
\[
    K(E_{\text e})
    = \sqrt{
        \dpd{N}{p_{\text e}} \frac{1}{p_{\text e}^2 F(p_{\text e}, Z)}
    }
    \propto
    \mathcal M_{\text i, \text f} \cdot (E_0 - E_{\text e}).
\]

Energie und Impuls können durch folgende dimensionslose Größen ausgedrückt
werden: \parencite[§1.5]{Guenther/K124}
\[
    \epsilon := \frac{E}{m_{\text e} c^2}
    \eqnsep
    \eta := \frac{p_{\text e}}{m_{\text e} c}.
\]

\begin{figure}[htbp]
    \centering
    \includegraphics[width=\linewidth]{../Fermifunktion.png}
    \caption{%
        Fermi-Funktion $F(Z, \epsilon)$ für den $\betaminus$-Übergang von
        Hf$^{181}$ (Kurve a), $\betaplus$-Übergang von Na$^{22}$ (Kurve b),
        $\betaminus$- und $\betaplus$-Übergang von Cu$^{64}$ (Kurve c bzw\@.
        d). Aus \parencite[Fig. 111]{Riezler/Kernphysikalisches} mit
        freundlicher Erlaubnis des Springer-Verlags.
    }
    \label{fig:fermifunktion}
\end{figure}

\section{Spektrometeraufbau}

Die beiden für diese Art Versuche interessanten Spektrometer sind das
Halbkreisspektrometer und das doppeltfokussierende $\piup \sqrt
2$-Spektrometer.

Das Halbkreisspektrometer arbeitet mit einem Permanentmagneten, der die
Elektronen auf einer Kreisbahn arbeitet. Jedoch werden Elektronen mit gleicher
Energie nicht auf ganz genau auf einen Punkt fokussiert, sondern in einem
Bereich $\Deltaup x$ verschmiert
\parencite[§2.231]{Riezler/Kernphysikalisches}. Je größer der Sollradius
$\rho_0$ ist, desto besser wird die Abbildung nach 
\[
    \Deltaup x = s + \frac{j^2}{\piup^2 \rho} + \rho \phi^2,
\]
wobei $s$ und $h$ die Breite und Höhe der Probe und $\phi$ der halbe
Öffnungswinkel der Strahlen aus der Probe sind
\parencite[(123)]{Riezler/Kernphysikalisches}.

Eine bessere Fokussierung schafft das doppeltfokussierende $\piup \sqrt
2$-Spektrometer durch eine inhomogene magnetische Flussdichte. Man nutzt dazu
eine rotationssymmetrische Flussdichte mit einem radialen Gradienten. In
\parencite[§2.232]{Riezler/Kernphysikalisches} wird nun folgende Herleitung der
benötigten Radialabhängigkeit $B(\rho)$ sowie des benötigten
Spektrometerwinkels $\Phi$ gegeben:

Sei der Sollkreis mit Radius $\rho_0$ durch den Elektronenimpuls $p$ durch $p =
e B \rho_0$ bestimmt. Ein Teilchen, dass in einem Winkel $\phi_\rho$ zum
Sollkreis von dessen Rand startet, schneidet ihn wieder am Rand um den Winkel
$\Phi_\rho$ versetzt. Für diesen Winkel gilt für kleine $\phi_\rho$:
\parencite[(125)]{Riezler/Kernphysikalisches}
\[
    \Phi_\rho = \piup \cdot \del{1 + \frac{\rho_0}{B(\rho_0)} \dpd B
    \rho}^{-\frac 12}.
\]

In der axialen Richtung gilt eine ähnliche Bedingung:
\parencite[(126)]{Riezler/Kernphysikalisches}
\[
    \Phi_z = \piup \cdot \del{- \frac{\rho_0}{B(\rho_0)} \dpd B
    \rho}^{-\frac 12}.
\]

Aus diesen beiden Gleichungen folgt laut den Autoren nun die Beziehung:
\parencite[(127)]{Riezler/Kernphysikalisches}
\[
    \frac1{\Phi_\rho^2} + \frac1{\Phi_z^2} = \frac1{\piup^2}.
\]

Mit der Bedingung, dass diese beiden Winkel gleich sein müssen ist die Lösung
der entsprechenden Differentialgleichung:
\parencite[(129)]{Riezler/Kernphysikalisches}
\[
    \Phi = \piup \sqrt 2 \approx \SI{256}{\degree}.
\]

Daher hat dieses Spektrometer seinen Namen.

Das Spektrometer hat noch eine Dispersion, so dass die Impulsintervalle nicht
gleich groß sind. Die Dispersion ist angegeben als:
\parencite[(130)]{Riezler/Kernphysikalisches}
\[
    \gamma = \frac{4 \rho_0}{B \rho},
\]
weshalb die Messwerte mit $\gamma$ zu multiplizieren sind. Da wir die Energie
mit den Konversionslinien eichen werden, ist jede zum Impuls proportionale
Größe geeignet \parencite[§P523.5.4]{physik512-Anleitung}.

\section{Detektoren}

\section{Hystere}

Eisen ist ferromagnetisch, es behält seine Magnetisierung $M$ auch ohne
angelegtes Magnetfeld $H$ bei. Legt man Eisen in ein Magnetfeld und erhöht die
Flussdichte bis zu einem Maximum (so dass die Magnetisierung in Sättigung geht)
und erniedrigt wieder bis $H = 0$, so nennt man die verbleibende Magnetisierung
\emph{Remanenzfeld}. Dieser Effekt ist isotrop, so dass man durch den
Mittelwert der beiden Remanenzfeldstärken die neutrale Mitte herausfinden kann.

Möchte man einen Ferromagneten entmagnetisieren, muss man das Magnetfeld, indem
sich der Ferromagnet befindet, in mehreren Durchläufen zwischen dem positiven
und negativen Maximum regeln, wobei ab dem zweiten Durchlauf der
Betrag der maximalen Magnetfeldstärke reduziert wird. Dadurch wird das
Remanenzfeld bei jedem Durchlauf schwächer, bis am Ende verschwindet. Diesen
Vorgang nennt man \emph{Hysterese}. Dieser Vorgang ist in
Abbildung~\ref{fig:hysterese} dargestellt.

\begin{figure}[htbp]
    \centering
    \begin{tikzpicture}
        \begin{axis}[
                width=\linewidth,
                height=0.7\linewidth,
                xlabel=Spulenstrom / wilk.\,Einh.,
                ylabel=Flussdichte / wilk.\,Einh.,
            ]
            \addplot[black] table [x index=1, y index=2]{../Daten/data.txt};
        \end{axis}
    \end{tikzpicture}
    \caption{%
        Hysteresekurve von Eisen. Messdaten aus \parencite{Ueding/248}.
    }
    \label{fig:hysterese}
\end{figure}

\chapter{Durchführung}

\chapter{Auswertung}


\IfFileExists{\bibliographyfile}{
    \printbibliography
}{}

\end{document}

% vim: spell spelllang=de tw=79
